\documentclass[10pt,a4paper]{article}
\usepackage[utf8]{inputenc}\DeclareUnicodeCharacter{2212}{-}
\usepackage{graphicx,wrapfig,lipsum}
\usepackage[T1]{fontenc}
\usepackage{float}
\usepackage[polish]{babel}
\usepackage{wrapfig} 
\usepackage{caption}
\usepackage{subcaption}
\usepackage{mathtools}
\usepackage{lmodern}
\usepackage{enumitem}
\usepackage{lscape}
\usepackage{url}
\usepackage{longtable}
\usepackage[version=4]{mhchem}
\usepackage{mathpazo}
\usepackage{booktabs}
\usepackage{pgf}

\author{Wojciech Noskowiak}
\title{Raport z ćwiczenia L9}
\date{Sierpień 2021}

\usepackage{natbib}
\usepackage{graphicx}

\begin{document}

\maketitle
\tableofcontents

\begin{abstract}
    Zbadałem stężenie dwutlenku azotu w próbkach powietrza pobranych z otoczeń różnych wyładowań elektrycznych w oparciu o metodę SSWO. W tym celu dopasowałem odpowiednie krzywe do uzyskanych eksperymentalnie danych. Większość uzyskanych przez mnie danych okazała się być miarodajna. Część uzyskanych przez mnie wyników okazała się być niezgodna z przewidywaniami teoretycznymi.
\end{abstract}

\newpage

\addcontentsline{toc}{section}{Wstęp}
\section*{Wstęp}
Celem ćwiczenia było zbadanie stężenia $\text{NO}_{\text{2}}$ w próbkach powietrza pobranych z otoczeń różnych wyładowań elektrycznych. Do tego celu wykorzystano w ćwiczeniu metodę SSWO. Wartości stężeń uzyskałem poprzez analizę dostarczonych mi wyniki pomiarów eksperymentalnych. Przekazane mi dane przebadałem w oparciu o polecenia z instrukcji \cite{instrukcja}, materiały dostępne na stronie pracowni \cite{strona} oraz dokumenty przekazane przez prowadzącego ćwiczenie \cite{SSWO}. Przekazane mi dane wpierw przeanalizowałem autorskim programem napisanym przeze mnie w języku python. Następnie otrzymane wartości  przepisałem do arkusza kalkulacyjnego przy pomocy którego wyliczyłem stężenia $\text{NO}_{\text{2}}$. Wyliczone stężenia wyraziłem w postaci cząstek na centymetr sześcienny oraz ppb (parts per bilion). Uzyskane wyniki przedstawiłem w tabeli.

\section{Wprowadzenie teoretyczne}

\subsection{Motywacja \cite{instrukcja}}

We współczesnej fizyce cząstek elementarnych powszechnie wykorzystywane są detektory gazowe. Dla otrzymania informacji o przechodzących przez taki detektor cząstkach neizbędna jest znajomość prędkości dryfu w wykorzystywanym niego gazie. Układy monitorujące prędkość dryfu stanowią więc integralną część wielu detektorów. 

\subsection{Pojęcia teoretyczne}

\subsubsection{Dryf}


\begin{equation*}
    \vec{j} = \sum_{k}{e_k n_k \vec{v_k}}
\end{equation*}

Gdzie indeks $k$ określa rodzaj cząstki naładowanej, a:
\begin{itemize}
    \item $e_k$ - ładunek danego  rodzaju cząstek
    \item $n_k$ - koncentracja danego rodzaju cząstek
    \item $v_k$ - prędkość dryfu cząstek danego rodzaju
\end{itemize}


\bibliographystyle{unsrt}
\bibliography{References}


\end{document}


